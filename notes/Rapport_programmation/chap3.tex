\chapter[Limites et évolutions de la solution]{Limites et évolutions de la solution}

\textit{La phase de programmation a permis de développer une première solution au sujet posé. Cependant, certaines fonctionnalités manquent encore. Ce chapitre détaille donc les améliorations pouvant être apportées au programme.}

\section{Limites de la solution proposée}

Bien que la solution proposée soit fonctionnelle et réponde à la problématique du projet, certaines fonctionnalités manquent encore :
\begin{itemize}[label=$\rightarrow$]
	\item Fonction de zoom : pour permettre à l'utilisateur d'agrandir ou de réduire l'entité présentée, une fonction de zoom peut être implémentée. Cette fonctionnalité pourra se baser sur les méthodes de la classe \textit{QGraphicsView}, et sur une fonction enregistrant les rotations de la molette de la souris.
	\item Gestion des marges trop grandes : dans la fenêtre de chargement, l'utilisateur peut définir des marges pour visualiser l'environnement des bâtiments. Pour l'instant, aucune fonctionnalité ne gère les cas de marges dépassant les bordures de l'orthoimage.
	\item Affichage des MNS : grâce aux méthodes de la classe \textit{QPixMap}, l'interface peut afficher une orthoimage couleur, constituée de trois bandes. Une nouvelle fonctionnalité pourrait gérer le cas de l'afficher des MNS constitués d'une seule bande. \\
\end{itemize}

De plus, l'implémentation de cette solution a été réalisée en s'appuyant sur un jeu de données types. Leur formalisme étant imposé à l'utilisateur, le programme devrait fonctionner avec d'autres données. Cependant, des tests complémentaires pourraient être réalisés pour vérifier la compatibilité de l'interface (utilisation d'une autre orthophoto, utilisation de plus d'emprises, ...).


\section{Évolution de la solution}

- Autres méthodes de filtrage\\
- Autre type de classification\\

