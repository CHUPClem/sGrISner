\chapter[Limites et évolutions de la solution]{Limites et évolutions de la solution}

\textit{La phase de programmation a permis de développer une première solution au sujet posé. Cependant, certaines fonctionnalités manquent encore. Ce chapitre détaille donc les améliorations pouvant être apportées au programme.}

\section{Limites de la solution proposée}

Bien que la solution proposée soit fonctionnelle et réponde à la problématique du projet, certaines fonctionnalités manquent encore :
\begin{itemize}[label=$\rightarrow$]
	\item Fonction de zoom -  Cette fonction a pour objectif de permettre à l'utilisateur d'agrandir ou de réduire l'entité présentée. Cette fonctionnalité pourra se baser sur les méthodes de la classe \textit{QGraphicsView}, et sur une fonction enregistrant les rotations de la molette de la souris.
	\item Gestion des marges trop grandes - Dans la fenêtre de chargement, l'utilisateur peut définir des marges pour visualiser l'environnement des bâtiments. Pour l'instant, aucune fonctionnalité ne gère les cas de marges dépassant les bordures de l'orthoimage.
	\item Affichage des MNS - Grâce aux méthodes de la classe \textit{QPixMap}, l'interface peut afficher une orthoimage couleur, constituée de trois bandes. Une nouvelle fonctionnalité pourrait gérer le cas de l'affichage des MNS constitués d'une seule bande. \\
\end{itemize}

L'implémentation de cette solution a été réalisée en s'appuyant sur un jeu de données types. Leur formalisme étant imposé à l'utilisateur, le programme devrait fonctionner avec d'autres données. Cependant, des tests complémentaires pourraient être réalisés pour vérifier la compatibilité de l'interface (utilisation d'une autre orthophoto, utilisation de plus d'emprises, ...).

\section{Évolution de la solution}

Comme nous venons de le voir, la solution détaillée précédemment peut nécessiter quelques améliorations techniques. Cependant, selon les besoins de l'utilisateur, il est aussi possible d'imaginer des évolutions plus conséquentes.\\

A ce titre, on pourrait définir de nouvelles méthodes de filtrage. Dans cette version du programme, deux méthodes ont été créées : la sélection \textit{Naïve} de toutes les entités, et la sélection \textit{Random} d'un nombre choisi d'entités. Comme défini lors de la phase d'analyse, d'autre stratégies peuvent être envisagées : présentation des entités ayant une faible probabilité, ou affichage des entités ayant des conflits de classification, ... \\

Le programme actuel se base sur une modélisation multi-classe du projet. Une autre évolution serait d'utiliser une modélisation multi-label. Dans ce cas, les données en entrée du programme aurait un formalisme différent qu'il faudrait prendre en compte dans l'implémentation.
