\chapter[Détail du code]{Détail du code}

\textit{Pour faciliter l'écriture et la compréhension du code, on utilise la méthode Modèle/Vue/Contrôleur. Dans cette section, on détaillera donc les différentes fonctions implémentées en suivant ce modèle.}

\section{Partie Vue}

La partie Vue contient le code relatif à l'interface graphique. Après la réalisation des trois interfaces graphiques avec QtDesigner, l'utilitaire PyUic5 a permis de transcrire le code C++ en langage Python. \\

\noindent Trois fichiers ont alors été créés et définissent les trois classes :
\begin{itemize}[label=$\rightarrow$]
	\item La classe \textit{Ui\_InterfacePrincipale} du fichier \textit{classificationActive.py} comporte le code de l'interface principale. Elle définit les caractéristiques des boutons, des labels de texte et de la fenêtre graphique dans une première méthode \textit{setupui} : ce sont les attributs de l'interface. Une méthode \textit{retranslateUi} permet de mettre à jour le texte des attributs.\\
	\item La classe \textit{Ui\_ChargementFichiers} du fichier \textit{chargementFichiers.py} comporte le code de l'interface de chargement. Elle dispose elle aussi des deux méthodes \textit{setupui} et \textit{retranslateUi}. Pour permettre l'affichage des différentes stratégies dans le menu déroulant, on utilise la variable globale STRATEGIES définie dans le fichier \textit{strategy.py}. Ainsi, ce menu sera mis à jours automatiquement en cas d'implémentation d'une nouvelle stratégie.\\
	\item La classe \textit{Ui\_ChoixClasse} du fichier \textit{choixClasse.py} comporte le code de l'interface de sélection d'une nouvelle classe. Elle dispose des méthodes \textit{setupui} et \textit{retranslateUi}.\\
\end{itemize}

En cas de modification de l'interface, ces fichiers sont réécrits, mais cela n'impactera pas le code du programme principal. Une seule modification doit être apportée à ces codes générés automatiquement : il faut modifier le chemin de l'image du point d'interrogation. En effet, ce chemin est contenue dans un fichier ressource en C++, qui n'est pas traduisible simplement en Python avec PyUic5. 

\section{Partie Modèle}

La partie Modèle contient organise le formalisme des données. Ainsi, trois nouveaux fichiers viennent définit trois classes : la classe \textit{Building}, la classe \textit{Background} et la classe \textit{Strategy}.\\

\subsection{La classe \textit{Building}}

La classe \textit{Building} est codée dans le fichier \textit{building.py}. Un objet de type \textit{Building} est défini par 4 attributs : identity (identifiant de l'objet), geometry, classe et probability. Deux méthodes lui sont associées : 
\begin{itemize}
	\item \textit{get\_points} retourne la liste de tous les sommets de chaque géométrie de l'entité. Ainsi, si une entité est composée de 3 triangles, la liste de sortie contiendra 9 points. Chaque point est codé par un couple de coordonnées (x,y).
	\item \textit{get\_bounding\_box} parcourt l'ensemble des coordonnées x et y d'une liste et retourne les valeurs maximales et minimales de ces deux paramètres. Cela permet de définir la fenêtre d'emprise d'une entité.
\end{itemize}


\section{Partie Contrôleur}