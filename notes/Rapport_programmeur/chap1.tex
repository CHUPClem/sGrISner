\chapter[Structure générale]{Structure générale}

\section{Configuration requise}

Le langage de programmation utilisé pour le code est le Python. La partie interface graphique utilise la bibliothèque Qt, et c'est la version 5 de PyQt qui a été utilisée.\\

\noindent D'autres packages de Python sont nécessaires au bon fonctionnement du programme :
\begin{itemize}
	\item gdal, permettant la gestion données géospatiales vectorielles ou raster ;
	\item pyshp, permettant l'extraction et la manipulation de Shapefile ;
	\item qimage2ndarray, une extension permettant la conversion entre les QImages et numpy.ndarray.\\
\end{itemize}

Le programme est multi-plateforme : il peut donc être ouvert sous Linux, Windows ou MAc OS.

\section{Données et formalisme}

Le programme considère plusieurs données en entrée, sélectionnées par l'utilisateur via une première interface :

\renewcommand{\arraystretch}{1.4}
\begin{figure}[!h]
	\begin{center}
		\begin{tabular}{|c|c|c|c|}
			\hline
			Données & Type de donnés & Correspondance dans le code\\
			\hline
			Classes d'erreurs possibles & Fichier .CSV & Variable de type dictionnaire\\
			Résultats de l'auto-qualification & Fichier .CSV & Liste de tuples\\
			Géométrie des entités & Dossier de .SHP & Attribut \textit{geometry} de la classe Bâtiment\\
			Orthophoto & Fichier .TIFF & Tuple (résolution + référence) + Matrice\\
			\hline
		\end{tabular}
	\end{center}
	\caption[Données en entrée]{Données en entrée}
	\label{tab:dataentre}
\end{figure}

\noindent De même, le formalisme des données en sortie est imposé :

\renewcommand{\arraystretch}{1.4}
\begin{figure}[!h]
	\begin{center}
		\begin{tabular}{|c|c|c|}
			\hline
			Données & Type de donnés & Correspondance dans le code \\
			\hline
			Résultats de l'interaction & Fichier .CSV & Liste de tuples \\
			\hline
		\end{tabular}
	\end{center}
	\caption[Données en sortie]{Données en sortie}
	\label{tab:datasortie}
\end{figure}
