\evenchapter[Analyse technique]{Analyse technique}

\textit{...}

\section{Description de l'existant}

\subsection{Programme de classification}
\begin{itemize}
	\item Objectif du programme de classification 
\end{itemize}

\subsection{Type de données et justification de leur utilisation}
\begin{itemize}
	\item CSV (lisible, ...)
	\item TIFF
	\item Shapefile (format vecto, ...)
\end{itemize}


\section{Choix techniques}

\subsection{Choix de développement}
\begin{itemize}
	\item Non choix de réaliser un plugin QGIS
	\begin{itemize}
		\item stratégie commandée par le commanditaire
		\item objectif de réaliser une interface modulaire qui ne dépend pas de QGIS (de ses évolution et choix technologiques)
		\item objectif d'utilisation à plus grande échelle de l'interface (IGN/MATIS/...) ce qui implique une flexibilité et un détachement de QGIS (et de tous ses modules qui ne sont pas forcément nécessaires pour notre cas)
	\end{itemize}
	\item Justification de l'utilisation de Qt
		\begin{itemize}
		\item détail des autres créateur d'interface possibles 
		\begin{itemize}
			\item WinForm/ Cocoa (spécifiques à un système d'exploitation)
			\item GTK (monde unix et portatif sous Windows mais pas natif - complexité de mise en oeuvre)
			\item HTML 5 + JavaScript (maintenance)
			\item wxWidget (moins utilisé que Qt)
		\end{itemize}
	\end{itemize}
	\item Justification de Python
	\begin{itemize}
		\item Possibilité de codage en C++ (+ performance) ou Python pour faire Qt
		\item Choix de python (simplicité de codage, multiplateforme)
	\end{itemize}
\end{itemize}
	
\subsection{Framework utilisé/Dépendance}
\begin{itemize}
	\item intégrations de librairies (pq ? Avantages/inconvénients ? Alternatives ?)
	\begin{itemize}
		\item Numpy
		\item Librairie pour rechercher/lire des fichiers
		\item Librairie de lecture des shapefile
	\end{itemize}
\end{itemize}

